\documentclass[twoside]{article}
\setlength{\oddsidemargin}{0.25 in}
\setlength{\evensidemargin}{-0.25 in}
\setlength{\topmargin}{-0.6 in}
\setlength{\textwidth}{6.5 in}
\setlength{\textheight}{8.5 in}
\setlength{\headsep}{0.75 in}
\setlength{\parindent}{0 in}
\setlength{\parskip}{0.1 in}

\usepackage{amsmath,amsfonts,graphicx}
\usepackage[hidelinks]{hyperref}
\usepackage{color}

%
% The following commands set up the lecnum (lecture number)
% counter and make various numbering schemes work relative
% to the lecture number.
%
\newcounter{lecnum}
\renewcommand{\thepage}{\thelecnum-\arabic{page}}
\renewcommand{\thesection}{\thelecnum.\arabic{section}}
\renewcommand{\theequation}{\thelecnum.\arabic{equation}}
\renewcommand{\thefigure}{\thelecnum.\arabic{figure}}
\renewcommand{\thetable}{\thelecnum.\arabic{table}}

%
% The following macro is used to generate the header.
%
\newcommand{\lecture}[1]{
   \pagestyle{myheadings}
   \thispagestyle{plain}
   \newpage
   \setcounter{lecnum}{1}
   \setcounter{page}{1}
   \noindent
   \begin{center}
   \framebox{
      \vbox{\vspace{2mm}
    \hbox to 6.28in { {\bf Nordic Graduate Course on Computational Mathematical Modeling
	\hfill August 2017} }
       \vspace{4mm}
       \hbox to 6.28in { {\Large \hfill #1  \hfill} }
       \vspace{2mm}
       {\it Authors: Miroslav Kuchta, Kent-Andre Mardal}
      \vspace{2mm}}
   }
   \end{center}
   \markboth{#1}{#1}

   These notes are meant as an accompanying text to the lecture on preconditioning
   given by the authors at the Nordic Graduate Course on Computational Mathematical Modeling
   held at Nordfjordeid in summer 2017. Computational exercises mentioned in the
   text are to be done using the IPython
   notebook\footnote{\url{https://github.com/MiroK/nordfjordeid/blob/master/preconditioning.ipynb}}.

   {\bf Disclaimer}: {\it
   These notes have not been subjected to the usual scrutiny reserved for formal
   publications.
   }
   \vspace*{4mm}
}

%Use this command for a figure; it puts a figure in wherever you want it.
%usage: \fig{NUMBER}{SPACE-IN-INCHES}{CAPTION}
\newcommand{\fig}[3]{
			\vspace{#2}
			\begin{center}
			Figure \thelecnum.#1:~#3
			\end{center}
	}

% Use these for theorems, lemmas, proofs, etc.
\newtheorem{theorem}{Theorem}[lecnum]
\newtheorem{lemma}[theorem]{Lemma}
\newtheorem{proposition}[theorem]{Proposition}
\newtheorem{claim}[theorem]{Claim}
\newtheorem{corollary}[theorem]{Corollary}
\newtheorem{definition}[theorem]{Definition}
\newenvironment{proof}{{\bf Proof:}}{\hfill\rule{2mm}{2mm}}

\newcommand*{\Scale}[2][4]{\scalebox{#1}{$#2$}}%
\newcommand*{\Resize}[2]{\resizebox{#1}{!}{$#2$}}%
\newcommand{\reals}{\mathbb{R}}
\newcommand{\sitewho}{\textcolor{magenta}{$\left[?!?\right]$}}
\newcommand{\site}[1]{\textcolor{magenta}{$\left[\text{#1}\right]$}}
\newcommand{\naturals}{\mathbb{N}}
\newcommand{\set}[1]{\{#1\}}
\newcommand{\abs}[1]{\lvert#1\rvert}
\newcommand{\semi}[1]{\lvert#1\rvert}
\newcommand{\norm}[1]{\lVert#1\rVert}
\newcommand{\nhalf}{\Scale[0.5]{-\tfrac{1}{2}}}
\newcommand{\dual}[1]{#1^{\prime}}
\newcommand{\Div}[1]{\nabla\cdot #1}
\renewcommand{\brack}[1]{\langle#1\rangle}
% ---Spaces---
\newcommand{\hone}{H^{1}\left(\Omega\right)}
\newcommand{\Hone}{\left[\hone]^3}
\newcommand{\Z}{Z}
\newcommand{\Zpolar}{Z^0}
\newcommand{\Zperp}{Z^\perp}
%% % ---Linalg---
\newcommand{\inv}[1]{\ensuremath{{#1}^{-1}}}
\newcommand{\tr}[1]{\ensuremath{{\text{tr}{#1}}}}
\newcommand{\transp}[1]{\ensuremath{{#1}^{\Scale[0.5]{\top}}}}
\newcommand{\la}[1]{\mathsf{#1}}
\DeclareMathOperator{\spn}{span}

% ---Change inf and sup alignments---
\makeatletter
\renewcommand{\inf}{\mathop{\@inf\vphantom{\@sup}}}
\renewcommand{\sup}{\mathop{\@sup\vphantom{\@inf}}}
\newcommand{\@inf}{\operatorname*{inf}}
\newcommand{\@sup}{\operatorname*{sup}}
\newcommand{\overbar}[1]{\mkern 1.5mu\overline{\mkern-1.5mu#1\mkern-1.5mu}\mkern 1.5mu}

\begin{document}

\lecture{Preconditioning}

\section{Introduction}
This lecture is about an important element of efficient solution algorithms for
solving linear systems $\la{A}\la{x}=\la{b}$. Here $\la{A}\in\reals^{n\times n}$
is a matrix while $\la{x}, \la{b}\in\reals^n$ are respectively the solution vector
and the right-hand side vector. The matrices we have in mind are large, $n \gg
10^5$, and will typically be sparse meaning that in each row there is only a small
(constant) number of nonzero entries. Consequently, to store the matrix / compute its 
matrix-vector product only $\mathcal{O}(n)$ numbers are needed.

One option to solve such systems is with direct solvers. Note that naive
implementation of Gaussian elimination requires $\mathcal{O}(\tfrac{2}{3}n^3)$
operations, e.g. \site{Quart}, and thus assuming $n=10^7$ and a computer performing $3\times 10^9$
operations per second the algorithm will compute the solution in more than six
thousand years. However, direct solvers implemented in UMFPACK,
SUPER\textunderscore{LU} or MUMPS (to name those that are interfaced from FEniCS)
are much more effcient. Taking into account the sparsity structure of the matrix
their complexity is bounded by $n z \log{n}$, see \site{super}, where $z$ is the number 
of nonzero entries. Still, solutions of systems with tens of millions of unknowns are not
attainable with these solvers.

Another option are iterative methods where the algorithms produce sequence of
vectors $\set{\la{x}_i}_{i=1}^{k}$ which should converge to the true solution $\la{x}$. We
would like an approximate solution $\la{x}_i$ to be produced at a cost of
$\mathcal{O}(n)$ operations (similar to the cost of matrix-vector product) and $k$
for which a good approaximation is obtained to be as small as possible. In order
for the latter we might instead of $\la{A}\la{x}=\la{b}$ consider the system
$\la{B}\la{A}\la{x}=\la{B}\la{b}$. Here $\la{B}$ is the preconditioner.

In standard texts such as \site{golub} or \site{strang} iterative methods are
first discussed for the system $\la{A}\la{x}=\la{b}$. Preconditioned versions of
the algorithms are only introduced later (in order to improve convergence
properties of the methods). In this sense, preconditioner is detached from the
linear system. Here we shall present things in a different way. The discrete systems 
we have in mind originate from discretization of partial differential equations and 
we will be primarily concerned with this continuous problem. We will see that the idea 
of preconditioner appears naturally as we discuss meaning of the solution /
well-posedness of the problem. Moreover, the ``continuous-first'' point of view will 
give us a template for designing preconditioners which lead to convergence in a number 
of iterations independent of the system size.


\section{The continuous problem}
The path that we have chosen to take begins with an \textit{abstract} problem into which the
\textit{concrete} continuous problems (equations), for which we shall design preconditioners later, 
can be recast. To present the problem, let us first fix the notation.

\subsection{Notation and preliminaries}
Let $V$ be a Hilbert space over some bounded domain $\Omega$. The space of bounded
linear functionals over $V$\footnote{Functional over $V$ is a mapping taking functions 
from $V$ and returning numbers, e.g. $v \mapsto \int_{\Omega} v(x)\,\mathrm{d}x$} 
shall be denoted $\dual{V}$. For $f\in\dual{V}$ the value of the functional in function 
$v\in V$ is $\brack{f, v}_{\dual{V}, V}$. We denote by $(\cdot, \cdot)_V$
the inner product of $V$ and $\norm{\cdot}_V$ shall be the corresponding norm,
i.e. $\norm{u}_V=\sqrt{(u, u)_V}$ for all $u\in V$. The norm of $\dual{V}$ is then 
$\norm{f}_{\dual{V}}=\sup_{v\in V}\tfrac{\semi{\brack{f, v}_{\dual{V},
V}}}{\norm{v}_V}$. Finally, we will use a shorthad notations for the norms and
inner products of $L^2(\Omega)$ and $H^1(\Omega)$ spaces
%
\[
  (u, v) = \int_{\Omega} u(x) v(x)\,\mathrm{d}x,
  \quad
  \norm{u} = \sqrt{(u, u)}
  \quad
  \mbox{ and }
  \quad
  \norm{u}_1 = \sqrt{\norm{u}^2 + \norm{\nabla{u}}^2}.
\]
%

Given $V$ with inner product $(\cdot, \cdot)_V$ suppose that $u\in V$ is fixed and
consider a mapping $f_u: v\rightarrow (u, v)_V$. Using Cauchy-Schwarz theorem\footnote{
  Recall that the angle $\theta$ of two vectors $\la{u}, \la{v} \in \reals^n$ is 
  defined as $\cos{\theta}=\frac{
    \semi{\la{u}\cdot {v}}
  }
  {
    \semi{\la{u}\cdot\la{u}}\semi{\la{v}\cdot\la{v}}
  }$. But then clearly
\[
  \semi{\la{u}\cdot\la{v}} 
  \leq \norm{\la{u}}\norm{\la{v}}
  \quad\quad
  \norm{\la{u}}=\sqrt{\la{u}\cdot\la{u}}
\]
holds for all vectors in $\reals^n$. Cauchy-Schwarz inequality is a generalization
of the statement above.}
%
see e.g. \site{brenner}, we have $\semi{(u, v)_V}\leq\norm{u}_V\norm{v}_V$ and so
$f_u\in \dual{V}$. Looking at $f_u$ we see that (some) elements of $\dual{V}$ can 
be manufactured via inner products. The \textit{Riesz map} theorem states that
\textit{all} the elements can be manufactured this way.

\begin{theorem}[Riesz]\label{thm:riesz}
  Given Hilbert space $V$ with an inner product $(\cdot, \cdot)_V$ there exists
  for every $f\in \dual{V}$ a unique function $\tau f\in V$ such that $\brack{f,
  v}_{\dual{V}, V}=(\tau f, v)_V$ for all $v\in V$. 
  Moreover, $\norm{f}_{dual{V}}=\norm{\tau f}_V$ (the Riesz map $\tau$ is an
  isometry).
\end{theorem}
\begin{proof}
See ?
\end{proof}

Note that in Theorem \ref{thm:riesz} we can consider the space $V$ with a different 
inner product and the choice will give rise to a different Riesz map. This is
illustrated in the next example, which is taken from \site{malek}. 

\begin{example}[Riesz maps for $H^1$]\label{ex:riesz}
  The space $V=H^1_0(\Omega)\subset H^1(\Omega)$ of $H^1$ functions taking zero values 
  on the boundary $\Omega$ is first considered with the inner product inducing the
  $\norm{\cdot}_1$ norm. For $f\in \dual{V}$ given, the definition of
  the Riesz map $\tau$ and the inner product then yields
  % 
  \[
    \brack{f, v}_{\dual{V}, V} = (\tau f, v)_V = (u, v) + (\nabla u, \nabla v)
    \quad \forall v\in V.
  \]
  %
  That is, the Riesz map with respect to this inner product produces $u=\tau f$
  where $u$ is the weak solution of the Helmholtz problem
  \[
    \begin{aligned}
      -\Delta u + u &= f\quad\mbox{ in }\Omega,\\
                  u &= 0\quad\mbox{ on }\partial\Omega.\\
    \end{aligned}
  \]

  We recall now that Poincar{\'e} inequality holds on $V$, i.e. there exists
  constant $C>0$ such that $\norm{u} \leq C \norm{\nabla u}$ for all $u\in V$. In
  turn $\norm{u}^2_1\leq(C^2+1)\norm{\nabla u}^2$ so that
  %
  \[
    (C^2+1)^{-1}\norm{\nabla}^2_1 \leq \norm{\nabla u}^2 \leq
    \norm{u}^2_1\quad\forall u \in V
  \]
  %
  and $u\mapsto \norm{\nabla u}$ thus defines a equivalent norm on $V$. If we next
  define a Riesz map with respect to the inner product inducing the new norm we
  obtain
  %
  \[
    \brack{f, v}_{\dual{V}, V} = (\nabla \tau f, \nabla v) \quad\forall v\in V.
  \]
  %
  The Riesz representation $u=\tau f$of $f$ is therefore a solution of the Poisson 
  problem
  \[
    \begin{aligned}
      -\Delta u &= f\quad\mbox{ in }\Omega,\\
                  u &= 0\quad\mbox{ on }\partial\Omega.\\
    \end{aligned}
  \]
\end{example}
  At this point we are ready to define an abstract problem and discuss its
  well-posedness.

%

\subsection{Existence}
  The abstract continuous problem shall be defined in terms of the bilinear form
  $a:V\times V\rightarrow \reals$ which is assumed to satisfy the following
  assumptions
  %
  \begin{subequations}\label{eq:lm_assume}
  \begin{align}
    \label{eq:contin}
    \mbox{There exists } C > 0 \mbox{ such that } \semi{a(u, v)\leq &C\norm{u}_V\norm{v}_V}
    \quad\forall u, v\in V, \\
    %%%%%
    \label{eq:coerc}
    \mbox{There exists }\alpha > 0 \mbox{ such that } \semi{a(u, u)\geq &\alpha\norm{u}^2_V}
    \quad\forall u\in V.
  \end{align}
  \end{subequations}
  %
  If \eqref{eq:lm_assume} hold we say that the bilinear form is continous and
  $V$-elliptic.

  Let now $u\in V$ be fixed and define $Au$ by $\brack{Au, v}_{\dual{V}, V}=a(u, v)$. 
  From the continuity condition \eqref{eq:contin} it follows that
  $Au\in\dual{V}$. However, we are interested in a more intriguing problem: given
  $b\in V$ we want to find $u\in V$ such that
  %
  \begin{equation}\label{eq:abstract}
    Au = b\quad\mbox{ in }\dual{V}
    \mbox{ or equivalently}
    \brack{Au, v}_{\dual{V}, V} = \brack{b, v}_{\dual{V}, V}\quad\forall v\in V.
  \end{equation}
  %
  % TODO: remark on special case of a symmetric
  Existence of solution of \eqref{eq:abstract} is the subject of Lax-Milgram
  theorem. In a special case when the bilinear form $a$ is symmetric\footnote{As a
  consequence we have 
  $\brack{Au, v}_{\dual{V}, V}=a(u, v)=a(v, u) = \brack{Av, u}_{\dual{V}, V}$} the
  existence follows from Theorem \ref{thm:riesz}. More specifically, if
  \eqref{eq:lm_assume} hold then due to symmetry of $a$ the bilinear form induces
  an inner product on $V$ and the representation theorem gives $u$ as $\tau b$
  where $\tau$ is the Riesz map with respect to the induced inner product. Note
  that in this case $\norm{u}_{V}=\norm{\tau}_{\dual{V}}$.

  % TODO: theorem+proof
  \begin{theorem}[Lax-Milgram]\label{thm:lm} Assume that \eqref{eq:lm_assume} hold.
  For any $b\in\dual{V}$ there exists a unique solution $u\in V$ of \eqref{eq:abstract}.
    Moreover $\norm{u}_V\leq \tfrac{1}{\alpha}\norm{b}_{\dual{V}}$.
  \end{theorem}
  \begin{proof}
  The following constructive proof can be found in \site{brenner}. We reproduce it here
  as it will be our first encounter with iterative algorithm (for solving \eqref{eq:abstract}).

  Let $u_0\in V$ be some guess for solution of \eqref{eq:abstract}. Since $b-Au=0$ for the
  exact solution it seems reasonable to build a procedure for improving the guess 
  using the residual $b-Au_0$. However, as the two are in different space, 
  $u_0\in V$ while $b-Au_0\in \dual{V}$, they cannot simply be added together. To
  transform the residual to the solution space a mapping the dual space is needed. 
  But we have already seen such a transformation! Using the Riesz map $\tau$ (Theorem
  \ref{thm:riesz} let us now define the iterative procedure for solving
  \eqref{eq:abstract}
  %
    \begin{equation}\label{eq:richardson}
      u_0 \leftarrow u_0 + \rho \tau (b-Au_0) =: T u_0
    \end{equation}
  %
  with $\rho$ a constant parameter to be suitably adjusted.
  
  Since $V$ is also a Banach space the proof that \eqref{eq:richardson}
  converges to a limit (solution) $u\in V$ uses Banach fixed point theorem. Thus
  contraction property of the mapping $T$ is needed: there exists $\epsilon < 1$
  such that
  %
  \[
    \norm{T(u_1 - u_2)}_V \leq \epsilon \norm{u_1 - u_2}_V \quad \forall u_1, u_2\in V.
  \]
  %
  This, however, follows from \eqref{eq:lm_assume} and properties of the Riesz
  map. Indeed letting $v=u_1-u_2$ we have
  %
    \[
  \begin{split}
    \norm{Tv}^2_V &= \norm{v-\rho\tau A v}^2_V = (v-\rho\tau A v, v-\rho\tau A v)_V\\
                &=\norm{v}^2_V - 2\rho(\tau A v, v)_V + \rho^2(\tau A v, \tau A v)_V\\
                &=\norm{v}^2_V - 2\rho\brack{A v, v}_V + \rho^2(\tau A v, \tau Av)_V
                \mathrlap{\textcolor{gray}{\text{ by} \brack{Au, v}_{\dual{V}, V}=a(u, v)}}\\
                &\leq \norm{v}^2_V - 2\rho\alpha\norm{v}^2_V + \rho^2(\tau A v, \tau Av)_V
                \mathrlap{\textcolor{gray}{\text{ by \eqref{eq:coerc}}}}\\
                &\leq \norm{v}^2_V - 2\rho\alpha\norm{v}^2_V + \rho^2\brack{A v, \tau Av)_{\dual{V}, V}}
                \mathrlap{\textcolor{gray}{\text{ by Riesz}}}\\
                &\leq \norm{v}^2_V - 2\rho\alpha\norm{v}^2_V + \rho^2 a(v, \tau Av)
                \mathrlap{\textcolor{gray}{\text{ by definition of } A}}\\
                &\leq \norm{v}^2_V - 2\rho\alpha\norm{v}^2_V + \rho^2 C
                \norm{v}_V\norm{\tau Av}_V
                \mathrlap{\textcolor{gray}{\text{ by \eqref{eq:contin}}}}\\
                &= \norm{v}^2_V - 2\rho\alpha\norm{v}^2_V + \rho^2 C \norm{v}_V\norm{Av}_V
                \mathrlap{\textcolor{gray}{\text{ by Riesz isometry }}}\\
                &\leq (1 - 2\rho\alpha + \rho^2 C^2)\norm{v}^2_V
                \mathrlap{\textcolor{gray}{\text{ by \eqref{eq:contin}}}}.\\
  \end{split}
    \]
  %
  To ensure that $(1 - 2\rho\alpha + \rho^2 C^2) < 1$ it now suffices to chose
  $\rho$ from the interval $(0, \tfrac{2\alpha}{C^2})$. Note that the smallest
  $\epsilon=1-\tfrac{\alpha^2}{C^2}$ is obtained for $C=\tfrac{\alpha}{C^2}$. Since
  the mapping $T$ is a contraction there exists a unique fixed point, $u=Tu$.

  Finally, the estimate of $u$ in terms of the right-hand side follows 
    from ellipticity of $a$ and the fact that $u$ solves \eqref{eq:abstract}
  \[
    \alpha\norm{u}^2_V \leq a(u, u) = \brack{A u, u}_{\dual{V}, V} = 
    \brack{f, u}_{\dual{V}, V} \leq \norm{f}_{\dual{V}}\norm{u}_V.
  \]
  \end{proof}
  
  % TODO: cond of operator
  %Following Lax-Milgram theorem we know that problem \eqref{eq:abstract} has
  %a unique solution and we can address convergence rate of the algorithm proposed
  %in the proof. Letting $u_k\in V$ denote the $k$-th vector computed by the
  %procedure we recall that the iterates are computed as $u_k = T u_{k-1}$ and the
  %solution $u$ of \eqref{eq:absract} satifies $u=Tu$. Letting now $e_k=u - u_k\in V$
  %be the error of $k$-th vector we observe that $e_k = T e_{k-1}$. Thus
  %$\norm{e_{k+1}}_V = \norm{T e_{k}} \leq \epsilon \norm{e_k}_V$ where $\epsilon$
  %is the contraction constant from the proof.
  %generated by algorithm \eqref{eq:richardson} and the error in the solution of
  %\eqref{eq:abstract} we see, cf. proof of Lax-Milgram theorem, that the norm of the 
  %error is governed by $\norm{Te_k}_V$
  

  % TODO: discretization

\end{document}
